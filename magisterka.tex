\documentclass[10pt,a4paper]{article}
\usepackage[utf8]{inputenc}
\usepackage[T1]{fontenc}
\usepackage{amsmath}
\usepackage{amsfonts}
\usepackage{amssymb}
\usepackage{graphicx}
\usepackage{indentfirst}
\usepackage{etoolbox}
\usepackage{placeins}
\usepackage{subcaption}

\usepackage[style=ieee, backend=bibtex ,sorting=none]{biblatex}
\addbibresource{refs.bib}
%opening
\author{Piotr Kumala}
\title{Mobilna aplikacja do wizualizacji pomiarów prowadzonych z pokładu drona}
\date{}

\renewcommand*\contentsname{Spis treści}
\renewcommand*\figurename{Rysunek}

\begin{document}
\begin{titlepage}
	\begin{center}
		\includegraphics[width=0.4\textwidth]{img/university.jpg}
		\vspace*{1cm}
		
		{\Huge
			\textbf{Praca magisterska} \\
			
		}
		\vspace{0.5cm}
		{\Large
			Zastosowanie rekurencyjnych sieci neuronowych do predykcji szeregów czasowych
			
			\vspace{1cm}
			
			\textbf{Piotr Kumala} \\
		}
		\vspace{0.5cm}
		kierunek studiów: \textbf{Informatyka Stosowana} \\
		
		\vspace{1cm}
		
		{\Large Opiekun: \textbf{dr hab. inż. Piotr A. Kowalski}}
		\vfill
		
		\
		\vspace{0.8cm}
		
		
		Wydział Fizyki i Informatyki Stosowanej\\
		Akademia Górniczo-Hutnicza im. Stanisława Staszica\\
		Kraków, lipiec 2021r.
		
	\end{center}
\end{titlepage}
\pagenumbering{gobble}
\newpage
\begin{center}
	\textbf{Oświadczenie studenta}
\end{center}

Uprzedzony o odpowiedzialności karnej na podstawie art. 115 ust. 1 i 2 ustawy z dnia
4 lutego 1994 r. o prawie autorskim i prawach pokrewnych (t.j. Dz. U. z 2018 r. poz. 1191 z
późn. zm.): „Kto przywłaszcza sobie autorstwo albo wprowadza w błąd, co do autorstwa całości
lub części cudzego utworu albo artystycznego wykonania, podlega grzywnie, karze
ograniczenia wolności albo pozbawienia wolności do lat 3. Tej samej karze podlega, kto
rozpowszechnia bez podania nazwiska lub pseudonimu twórcy cudzy utwór w wersji
oryginalnej albo w postaci opracowania, artystyczne wykonanie albo publicznie zniekształca
taki utwór, artystyczne wykonanie, fonogram, wideogram lub nadanie.”, a także uprzedzony o
odpowiedzialności dyscyplinarnej na podstawie art. 307 ust. 1 ustawy z dnia 20 lipca 2018r.
Prawo o szkolnictwie wyższym i nauce (Dz. U. z 2018 r. poz. 1668 z późn. zm.) „Student
podlega odpowiedzialności dyscyplinarnej za naruszenie przepisów obowiązujących w uczelni
oraz za czyn uchybiający godności studenta.”, oświadczam, że niniejszą pracę dyplomową
wykonałem osobiście i samodzielnie i nie korzystałem ze źródeł innych niż wymienione w
pracy. 

Jednocześnie Uczelnia informuję, że zgodnie z art. 15a ww. ustawy o prawie autorskim
i prawach pokrewnych Uczelni przysługuje pierwszeństwo w opublikowaniu pracy
dyplomowej studenta. Jeżeli Uczelnia nie opublikowała pracy dyplomowej w terminie 6
miesięcy od dnia jej obrony, autor może ją opublikować chyba, że praca jest częścią utworu
zbiorowego. Ponadto Uczelnia, jako podmiot, o którym mowa w art. 7 ust. 1 pkt 1 ustawy z
dnia 20 lipca 2018 r. — Prawo o szkolnictwie wyższym i nauce (Dz. U. z 2018 r. poz. 1668 z
późn. zm.), może korzystać bez wynagrodzenia i bez konieczności uzyskania zgody autora z
utworu stworzonego przez studenta w wyniku wykonywania obowiązków związanych z
odbywaniem studiów, udostępnić utwór ministrowi właściwemu do spraw szkolnictwa
wyższego i nauki oraz korzystać z utworów znajdujących się w prowadzonych przez niego
bazach danych, w celu sprawdzania z wykorzystaniem systemu antyplagiatowego. Minister
właściwy do spraw szkolnictwa wyższego i nauki może korzystać z prac dyplomowych
znajdujących się w prowadzonych przez niego bazach danych w zakresie niezbędnym do
zapewnienia prawidłowego utrzymania i rozwoju tych baz oraz współpracujących z nimi
systemów informatycznych.

\vfill
\
\begin{flushright}
	............................................................... \\
	(czytelny podpis)
\end{flushright}

\newpage
\pagenumbering{arabic}
\tableofcontents
\newpage

\section{Wprowadzenie i cel pracy}

\subsection{Wprowadzenie}
Sieci neuronowe są coraz powszechniej używane do rozwiązywania wielu problemów życia codziennego. Nawet w wydawałoby się prostych urządzeniach jak telefony zawierane są układy elektroniczne specjalizujące się w obliczeniach korzystających z sieci neuronowych \cite{appleNeuralEngine}. Jednym z najczęściej spotykanych problemów w życiu codziennym jest próba predykcji szeregu czasowego warto więc spróbować wykorzystać sieci neuronowe do jego rozwiązania. Do tego celu zastosować można różne rodzaje sieci neuronowych ja chciałbym się skupić na zastosowaniu rekurencyjnych sieci neuronowych.

\subsection{Cel pracy}
Rekurencyjne sieci neuronowe są szeroko wykorzystywane i rozwijane przez największe firmy technologiczne. Są one stosowane między innymi do rozpoznawania pisma odręcznego i tekstu dyktowanego (Google Android LSTM). Mając na uwadze ilość wysiłku wkładanego w ulepszanie tychże sieci warto sprawdzić jak radzą one sobie w nieco bardziej nietypowych dla nich dziedzinach. W tej pracy chcę przedstawić badanie użyteczności zastosowania rekurencyjnych sieci neuronowych do predykcji szeregów czasowych. W przypadku wykazania, że są one użyteczne do rozwiązywania problemów tego typu dostaniemy "za darmo" nowy dobry sposób predykcji. 

\section{Wstęp teoretyczny}
\subsection{Szereg czasowy}
\subsection{Regresja liniowa}
\subsection{Drzewa decyzyjne}
\subsection{Rekurencyjne sieci neuronowe}
\subsection{Architektura LSTM}
\subsection{Architektura GRU}

\section{Przygotowanie zbioru danych}
\section{Przeprowadzone badania}
\section{Otrzymane wyniki}
\section{Wnioski}

\printbibliography

\end{document}