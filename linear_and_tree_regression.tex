\documentclass{article}
\usepackage{polski}
\usepackage[utf8]{inputenc}
\usepackage{graphicx}
\usepackage{placeins}
\graphicspath{ {./img/} }

\title{Regresja liniowa i drzewa decyzyjne}
\author{Piotr Kumala}
\date{06.06.2022 r.}

\begin{document}
	
	\maketitle
	
	\section{Wstęp}
	W tym dokumencie przedstawione zostanie zastosowanie klasycznych metod predykcji szeregów czasowych, ich wyniki oraz wnioski z nich płynące. Uzyskane wyniki zostaną wykorzystane jako wartości referencyjne przy analizie skuteczności zastosowań rekurencyjnych sieci neuronowych do tego samego problemu. W podsumowaniu przedstawiony zostanie również dalszy kierunek prowadzonych prac oraz ostateczny cel tychże badań. 
	
	\section{Wybór danych do dalszej analizy}
	W trakcie przeprowadzania regresji liniowej zauważono, że przygotowany wcześniej zbiór danych o wypadkach rowerowych w Wielkiej Brytanii niestety nie jest odpowiedni do przeprowadzania na nim predykcji. Każdy wypadek posiada informacje o warunkach pogodowych, oświetleniu, porze dnia i rodzaju drogi. Nie można jednak na podstawie żadnej z tych charakterystyk prowadzić wnioskowania o ilości wypadków danego dnia. Nie został przygotowany szereg czasowy opisujący te cechy, a stworzenie go jest bardzo skomplikowane o ile w ogóle wykonalne. 
	Z tego powodu podjęta została decyzja o porzuceniu tych danych i skupieniu dalszej pracy nad danymi klimatyczno-pogodowymi Krakowa w latach 2000-2021. W pracy nad tymi danymi nie zauważono żadnych znaczących problemów i wydaje się, że są one odpowiednie do prowadzenia dalszych badań. 
	
	\section{Regresja liniowa}
	\section{Drzewo decyzyjne}
	\section{Wnioski}
	
\end{document}